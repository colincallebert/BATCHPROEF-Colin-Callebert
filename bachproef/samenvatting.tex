%%=============================================================================
%% Samenvatting
%%=============================================================================

% TODO: De "abstract" of samenvatting is een kernachtige (~ 1 blz. voor een
% thesis) synthese van het document.
%
% Een goede abstract biedt een kernachtig antwoord op volgende vragen:
%
% 1. Waarover gaat de bachelorproef?
% 2. Waarom heb je er over geschreven?
% 3. Hoe heb je het onderzoek uitgevoerd?
% 4. Wat waren de resultaten? Wat blijkt uit je onderzoek?
% 5. Wat betekenen je resultaten? Wat is de relevantie voor het werkveld?
%
% Daarom bestaat een abstract uit volgende componenten:
%
% - inleiding + kaderen thema
% - probleemstelling
% - (centrale) onderzoeksvraag
% - onderzoeksdoelstelling
% - methodologie
% - resultaten (beperk tot de belangrijkste, relevant voor de onderzoeksvraag)
% - conclusies, aanbevelingen, beperkingen
%
% LET OP! Een samenvatting is GEEN voorwoord!

%%---------- Nederlandse samenvatting -----------------------------------------
%
% TODO: Als je je bachelorproef in het Engels schrijft, moet je eerst een
% Nederlandse samenvatting invoegen. Haal daarvoor onderstaande code uit
% commentaar.
% Wie zijn bachelorproef in het Nederlands schrijft, kan dit negeren, de inhoud
% wordt niet in het document ingevoegd.

\IfLanguageName{english}{%
\selectlanguage{dutch}
\chapter*{Samenvatting}
\selectlanguage{english}
}{}

%%---------- Samenvatting -----------------------------------------------------
% De samenvatting in de hoofdtaal van het document

\chapter*{\IfLanguageName{dutch}{Samenvatting}{Abstract}}
Bij een monolithische architectuur wordt de volledige applicatie als één enkel geheel met een eigen codebase en infrastructuur ontwikkeld. In een microservicesarchitectuur wordt de functionaliteit opgesplitst in kleinere, zelfstandig werkende services, die elk hun eigen codebase en infrastructuur gebruiken. Het is cruciaal om bij het kiezen van een architectuur rekening te houden met de specifieke behoeften van de applicatie en de bedrijfsdoelstellingen. Microservices kunnen de ontwikkelingstijd versnellen en de complexiteit van de applicatie verminderen, terwijl een monolithische architectuur beter geschikt kan zijn voor minder complexe applicaties die minder onderhoud vereisen.


Bij het implementeren van microservices zijn aspecten zoals load balancing, service discovery, netwerkbeveiliging en weerbaarheid essentieel om de schaalbaarheid en prestaties te garanderen. Servicemeshes bieden functionaliteit om deze uitdagingen aan te pakken. Bij het opzetten van een servicemesh voor bedrijfsapplicaties is het belangrijk om architecturale overwegingen en best practices te volgen om optimale schaalbaarheid en prestaties te bereiken.


Dit onderzoek richt zich op het bepalen van de geschikte architectuurkeuze voor een specifieke situatie, aan de hand van een case study. Hierbij worden de voor- en nadelen van zowel monolithische als microservicesarchitecturen belicht. Eerst wordt een monolithische applicatie opgezet, waarna deze wordt opgesplitst in microservices. De prestaties van beide architecturen worden vervolgens getest en geanalyseerd. De resultaten van dit onderzoek kunnen bijdragen aan het optimaliseren van applicatieontwikkeling en -prestaties.


Deze bachelorproef biedt een vergelijking tussen monolithische en microservicesarchitecturen, met een focus op de praktische toepassingen en de impact op bedrijfsapplicaties. Door de resultaten van de case study en de opgedane inzichten kunnen organisaties beter geïnformeerde beslissingen nemen over welke architectuur het beste aansluit bij hun specifieke behoeften en doelstellingen. 


In deze context onderzochten we de verschillende aspecten van beide architecturen, waaronder de operationele complexiteit, schaalbaarheid, onderhoudbaarheid en de prestaties. De bevindingen uit dit onderzoek geven een duidelijk beeld van de voordelen en uitdagingen van het gebruik van microservices in vergelijking met een monolithische benadering. Dit kan organisaties helpen om strategische keuzes te maken die hun IT-infrastructuur en applicatiebeheer verbeteren.


De samenvatting geeft een overzicht van de belangrijkste bevindingen en aanbevelingen die voortkomen uit het onderzoek. Het biedt waardevolle inzichten voor ontwikkelaars, IT-beheerders en besluitvormers die overwegen om de transitie naar een microservicesarchitectuur te maken. De resultaten tonen aan dat, hoewel de overstap naar microservices aanzienlijke voordelen biedt, er ook uitdagingen zijn die zorgvuldig moeten worden beheerd om de voordelen volledig te kunnen benutten.
