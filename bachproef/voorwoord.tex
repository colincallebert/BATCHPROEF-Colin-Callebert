%%=============================================================================
%% Voorwoord
%%=============================================================================

\chapter*{\IfLanguageName{dutch}{Woord vooraf}{Preface}}%
\label{ch:voorwoord}

%% TODO:
%% Het voorwoord is het enige deel van de bachelorproef waar je vanuit je
%% eigen standpunt (``ik-vorm'') mag schrijven. Je kan hier bv. motiveren
%% waarom jij het onderwerp wil bespreken.
%% Vergeet ook niet te bedanken wie je geholpen/gesteund/... heeft
In deze bachelorproef onderzoek ik de voor- en nadelen van microservices ten opzichte van monolithische applicaties. Dit onderwerp sprak mij om verschillende redenen aan. Ten eerste ben ik geïnteresseerd in het ontwikkelen van applicaties en het optimaliseren van hun prestaties. Ten tweede ben ik ook gefascineerd door de verschillende architectuurkeuzes die mogelijk zijn bij het ontwikkelen van applicaties. Ten slotte vind ik het verkennen en leren van nieuwe technologieën een boeiende en plezierige ervaring.


Tijdens mijn stage bij Cloudway, een bedrijf gericht op de cloud, draaide mijn project volledig in de cloud en maakte het gebruik van microservices. Deze praktijkervaring gaf mij diepgaande inzichten in hoe microservices in de praktijk worden toegepast en welke uitdagingen en voordelen ze met zich meebrengen. Het stelde mij in staat om theoretische kennis direct toe te passen en te testen in een professionele omgeving, wat mijn begrip en interesse in dit onderwerp verder versterkte.


Graag wil ik mijn co-promotor, David Vanden Bussche, bedanken voor zijn begeleiding en waardevolle feedback gedurende het schrijven van deze bachelorproef. Zijn expertise en geduld hebben een cruciale rol gespeeld in het succesvol afronden van dit onderzoek. 


Tot slot wil ik mijn familie en vrienden bedanken voor hun onvoorwaardelijke steun en motivatie gedurende deze periode. Hun aanmoedigingen en begrip hebben mij geholpen om dit onderzoek met succes af te ronden. Ik hoop dat deze bachelorproef een nuttige bron van informatie zal zijn voor studenten, ontwikkelaars en IT-beheerders die meer willen weten over de impact van microservices op de ontwikkeling en het beheer van bedrijfsapplicaties. Ik wens u veel leesplezier en hoop dat de inzichten en best practices die in dit werk worden gepresenteerd, waardevol zullen blijken in uw eigen professionele praktijk.
